\usepackage[utf8]{inputenc}
\DeclareUnicodeCharacter{00A0}{ }%Permet d'éviter certains conflits de caractères invisibles
\usepackage{amssymb}            % Principaux symboles
%\usepackage{fontspec}
%\usepackage{xunicode}
%\usepackage{xltxtra}
\usepackage[frenchb]{babel}
%\defaultfontfeatures{Scale=MatchLowercase}
%\setmainfont[Mapping=tex-text,Ligatures={Common, Historical}]{Linux Libertine O}
%\setsansfont[Mapping=tex-text]{Linux Biolinum O}
%\setmonofont[Scale=0.75]{DejaVu Sans Mono}

%% Packages pour le texte
\usepackage[misc,geometry]{ifsym}	% Police numéros battons
\usepackage{pifont}		% Police \ding
\usepackage{eurosym}		% Symbole de l'euro
\usepackage{soul}		% Souligner
\usepackage{enumerate}		% Listes
\usepackage{verbatim}		% Codes source
\usepackage{moreverb}		%	et listings
\usepackage{textcomp}
\usepackage{multicol}

%% Packages pour les tableaux
\usepackage{array}		% Outils supplémentaires
\usepackage{multirow}		% Colonnes multiples
\usepackage{tabularx}		% Largeur totale donnée
\usepackage{longtable}		% Sur plusieurs pages

%% Les packages pour les dessins
\usepackage{graphicx}		% Insertion de figures
%\usepackage{picins}		% Dans un paragraphe
\usepackage{epic}		% Capacités graphiques
\usepackage{eepic}		% 	étendues
\usepackage{afterpage}		% Voir page 69
\usepackage{rotating}		% Tourner du texte
\usepackage{caption}		% Légendes
% \addto\captionsfrench{\def\figurename{}}

%% Packages pour les maths
\usepackage{amsmath}		% Commandes essentielles
\usepackage{amssymb}		% Principaux symboles
\usepackage{mathrsfs}		% Police calligraphique
\usepackage{theorem}		% Théorèmes
%\usepackage{tikz}		% Courbes
\usepackage{esvect}             % Vecteurs
%\usetikzlibrary{shapes,arrows,shadows}
\usepackage{pgf}
%\usetikzlibrary{arrows}
% Packages pour la physique
%\usepackage{sistyle}		% Unités
\usepackage[version=3]{mhchem}	% Formules chimiques
\usepackage{etex}
\usepackage{m-pictex,m-ch-en}

%\usepackage{media9}
\usepackage{multimedia}		% Vidéos dans la présentation
%\usepackage{movie15}

\usepackage{ccicons}		% Licence creativecommons

%\SIdecimalsign{,}


\AtBeginSection[]
{
  \begin{frame}
    \frametitle{Sommaire}
    \begin{multicols}{2}
      {\small
        \tableofcontents[currentsection, hideothersubsections]}
    \end{multicols}
  \end{frame}
}

\usetheme{Warsaw}


